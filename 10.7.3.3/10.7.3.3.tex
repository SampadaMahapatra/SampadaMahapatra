\documentclass{article}
\usepackage{amsmath}
\newcommand{\myvec}[1]{\ensuremath{\begin{pmatrix}#1\end{pmatrix}}}
\newcommand{\mydet}[1]{\ensuremath{\begin{vmatrix}#1\end{vmatrix}}}
\newcommand{\solution}{\noindent \textbf{Solution: }}
\providecommand{\brak}[1]{\ensuremath{\left(#1\right)}}
%\providecommand{\norm}[1]{\left\lVert#1\right\rVert}
\let\vec\mathbf
\title{Ch - 7 Coordinate geometry}
\author{Mahapatra Sampada(mahapatra.sampada@sriprakashschools.com)}
\date{1 August 2023}
\begin{document}
\maketitle
\section*{Class10${th}$ Maths- chapter 7}
This is problem 3 of exercise 7.3
\begin{enumerate}
\item Find the area of a triangle formed by joining the mid points of the sides of the triangle whose vertices are (0,-1) , (2,1) and (0,3). Find the ratio of this area to the area of the given triangle. \\

\solution\\
Let the points be A(0,-1) , B(2,1) , C (0,3)\\ 
Hence the points are :\\
D = (1,0)\\
E = (1,2)\\
F = (0,1)\\
    Area of triangle = 
\begin{align}
    [\frac{1}{2}][\lvert x_1(y_2 - y_3)+x_2(y_3 - y_1)+x_3(y_1 - y_2)\rvert]
\end{align}
From this formula, area of triangle ABC=
 \begin{align}
 [\frac{1}{2}] [\lvert  0(1-3)+ 2(3+1)+ 0(-1-1)\rvert] =&
 4sq.unit
 \end{align}
Likewise, the area of the triangle DEF= 
 \begin{align}
 [\frac{1}{2}] [\lvert  1(2-1)+ 1(1-0)+ 0(0-2)\rvert] =&
1 sq.unit
 \end{align}
 
 Therefore, the ratio between the triangle DEF and ABC = 1:4
\end{enumerate}
\end{document}